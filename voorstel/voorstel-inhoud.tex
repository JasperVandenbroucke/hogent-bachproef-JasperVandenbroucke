%---------- Inleiding ---------------------------------------------------------

% TODO: Is dit voorstel gebaseerd op een paper van Research Methods die je
% vorig jaar hebt ingediend? Heb je daarbij eventueel samengewerkt met een
% andere student?
% Zo ja, haal dan de tekst hieronder uit commentaar en pas aan.

%\paragraph{Opmerking}

% Dit voorstel is gebaseerd op het onderzoeksvoorstel dat werd geschreven in het
% kader van het vak Research Methods dat ik (vorig/dit) academiejaar heb
% uitgewerkt (met medesturent VOORNAAM NAAM als mede-auteur).
% 

\section{Inleiding}%
\label{sec:inleiding}

Waarover zal je bachelorproef gaan? Introduceer het thema en zorg dat volgende zaken zeker duidelijk aanwezig zijn:

\begin{itemize}
  \item kaderen thema
  \item de doelgroep
  \item de probleemstelling en (centrale) onderzoeksvraag
  \item de onderzoeksdoelstelling
\end{itemize}

Denk er aan: een typische bachelorproef is \textit{toegepast onderzoek}, wat betekent dat je start vanuit een concrete probleemsituatie in bedrijfscontext, een \textbf{casus}. Het is belangrijk om je onderwerp goed af te bakenen: je gaat voor die \textit{ene specifieke probleemsituatie} op zoek naar een goede oplossing, op basis van de huidige kennis in het vakgebied.

De doelgroep moet ook concreet en duidelijk zijn, dus geen algemene of vaag gedefinieerde groepen zoals \emph{bedrijven}, \emph{developers}, \emph{Vlamingen}, enz. Je richt je in elk geval op it-professionals, een bachelorproef is geen populariserende tekst. Eén specifiek bedrijf (die te maken hebben met een concrete probleemsituatie) is dus beter dan \emph{bedrijven} in het algemeen.

Formuleer duidelijk de onderzoeksvraag! De begeleiders lezen nog steeds te veel voorstellen waarin we geen onderzoeksvraag terugvinden.

Schrijf ook iets over de doelstelling. Wat zie je als het concrete eindresultaat van je onderzoek, naast de uitgeschreven scriptie? Is het een proof-of-concept, een rapport met aanbevelingen, \ldots Met welk eindresultaat kan je je bachelorproef als een succes beschouwen?

%---------- Stand van zaken ---------------------------------------------------

\section{Literatuurstudie}%
\label{sec:literatuurstudie}

Het overschakkelen van een monolitische naar een microservices-architectuur webapplicatie is een complex proces. Daarom is het belangrijk dat ontwikkelaars kennis kunnen opdoen van hoe een microservices-architectuur tot stand komt en wat er allemaal bij komt kijken.

\subsection{Monolitische architectuur}

Allereerst wordt er gekeken naar wat een monolitische architectuur nu juist is. Volgens \textcite{Velepucha2023} is een monolitische architectuur gekenmerkt door het hebben van slechts 1 enkel startpunt. Daarnaast vermelden \textcite{Velepucha2023} dat een monolitische applicatie doorgaans gekenmerkt wordt door verschillende logische lagen. Een referentie architectuur hiervoor is bijvoorbeeld het 3 lagen patroon \autocite{Velepucha2023}. Hierbij wordt de monolitische applicatie gebouwd in 3 lagen die elk hun verantwoordelijkheid dragen, namelijk de presentatie laag, de domein laag en de data laag.

\textcite{Blinowski2022} verduidelijken de functie van elk van deze lagen. In de presentatie laag wordt er voornamelijk functionaliteit geplaatst die het voor de gebruiker mogelijk maakt om met de applicatie te interageren. Vervolgens wordt de domein laag gebruikt om te reageren om verzoeken van de gebruiker. Dit kan gaan van het uitvoeren van bepaalde logica tot het ophalen van data uit een databron.

\subsection{Microservices-architectuur}

\textcite{Thoenes2015} beschrijft een microservice als een kleine applicatie met één enkele verantwoordelijkheid, die individueel uigerold, geschaald en getest kan worden. Het feit dat een microservice slechts één verantwoordelijkheid heeft kan in twee opzichten bekeken worden \autocite{Thoenes2015}. Ten eerste houdt dit in dat een microservice slechts één reden heeft om te worden gewijzigd of vervangen. En ten tweede betekent deze verantwoordelijkheid dat een microservice slechts één taak uitvoert, die ook nog eens eenduidig is.

De microservices-architectuur is een software architectuur waarbij één enkele applicatie wordt gebouwd die bestaat uit verschillende, kleinere microservices \autocite{Lewis2014}. Deze kleinere services kunnen onafhankelijk van elkaar werken en communiceren met elkaar door gebruik te maken van bijvoorbeeld Hypertext Transfer Protocol (HTTP) of een Application Programming Interface (API).

% Voor literatuurverwijzingen zijn er twee belangrijke commando's:
% \autocite{KEY} => (Auteur, jaartal) Gebruik dit als de naam van de auteur
%   geen onderdeel is van de zin.
% \textcite{KEY} => Auteur (jaartal)  Gebruik dit als de auteursnaam wel een
%   functie heeft in de zin (bv. ``Uit onderzoek door Doll & Hill (1954) bleek
%   ...'')

%---------- Methodologie ------------------------------------------------------

\section{Methodologie}%
\label{sec:methodologie}

In de initiële fase van het onderzoek zal er een uitgebreide literatuurstudie worden uitgevoerd. De bedoeling van deze fase is om inzicht te krijgen in het onderwerp en het probleemdomein van deze bachelorproef.

In een volgende fase zal er onderzoek worden gedaan naar de meest toepasselijke programmeertalen en frameworks om een proof-of-concept (PoC) uit te werken. Hierbij wordt er ook aandacht gezonken aan containerisatietools, namelijk Docker en Kubernetes, die het mogelijk maken om microservices onafhankelijk te ontwikkelen, testen en te beheren.

In de derde fase van het onderzoek wordt een effectieve PoC uitgewerkt. Voor deze PoC wordt er beroep gedaan op de technologieën die het best uit literatuurstudie naar voorkomen. In een eerste stap van het uitwerken van deze PoC, zal een monolitische e-commerce webapplicatie gebouwd worden. In een volgende stap wordt diezelfde webapplicatie omgezet naar een microservices webapplicatie. Hierbij zal er rekeninggehouden worden met de best-practices voor het ontwerpen van een microservices-architectuur.

In de daaropvolgende fase zullen de schaalbaarheid en de complexiteit van beide webapplicaties getest worden. Voor de schaarbaarheidstesten zal er gebruik gemaakt worden van Apache JMeter. Deze gratis open-source software maakt het mogelijk om simulaties van hoge belasting uit te voeren, zodat de prestanties onder intesief gebruik van de beide applicaties kunnen vergeleken worden. Daarnaast wordt de complexiteit beoordeeld. Hiervoor zal de focus gelegd worden op de ontwikkelings- en onderhoudscomplexiteit van beide applicaties.

Tot slot zullen de resultaten van de Apache JMeter testen en de complexiteitsbeoordeling van beide webapplicaties worden geanalyseerd om conclusies te trekken over de schaalbaarheid en complexiteit van de microservices-architectuur ten opzichte van de monolitische aanpak.

%---------- Verwachte resultaten ----------------------------------------------
\section{Verwacht resultaat, conclusie}%
\label{sec:verwachte_resultaten}

Na het ontwikkelen van de PoC en het uitvoeren van de testen wordt verwacht dat de resultaten van de schaalbaarheidstesten duidelijk zullen aantonen dat doormiddel van een microservices-architectuur de webapplicatie beter presteert bij hoge belasting. Tegelijkertijd zal het hoogstwaarschijnlijk ook blijken dat het omzetten van een monolitische webapplicatie naar een microservices-architectuur heel wat uitdagingen met zich meebrengt. De resultaten zullen worden weergegeven met behulp van grafieken gegenereerd door Apache JMeter. Daarnaast wordt verwacht dat de webapplicatie met de microservices-architectuur een hogere ontwikkel- en onderhoudscomplexiteit met zich meebrengt ten opzichte van de monolitische webapplicatie.

Voor de ontwikkelaars bij Fabrimode is de meerwaarde te vinden in de inzichten en resultaten die deze bachelorproef te bieden heeft. Deze bevindingen kunnen door Fabrimode in acht genomen worden wanneer zij beslissen om al dan niet een project op te starten die de overgang naar een microservices-architectuur uitvoert.

