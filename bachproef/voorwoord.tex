%%=============================================================================
%% Voorwoord
%%=============================================================================

\chapter*{\IfLanguageName{dutch}{Woord vooraf}{Preface}}%
\label{ch:voorwoord}

%% TODO:
%% Het voorwoord is het enige deel van de bachelorproef waar je vanuit je
%% eigen standpunt (``ik-vorm'') mag schrijven. Je kan hier bv. motiveren
%% waarom jij het onderwerp wil bespreken.
%% Vergeet ook niet te bedanken wie je geholpen/gesteund/... heeft

Met trots presenteer ik u mijn bachelorproef, "Optimalisatie van webapplicaties met behulp van microservices-architectuur bij stagebedrijf Fabrimode: een onderzoek naar schaalbaarheid en complexiteit.", geschreven ter afronding van mijn opleiding Bachelor in de Toegepaste Informatica aan de Hogeschool Gent. Deze bachelorproef is het sluitstuk van een intensieve, maar vooral enorm verrijkende periode van drie jaar. Toen ik aan deze studie begon, beschikte ik over nagenoeg geen programmeerkennis. Vandaag de dag, aan het einde van deze opleiding, ben ik bijzonder trots op de vaardigheden die ik heb verworven. Deze transformatie van complete beginner tot een programmeur met een solide fundament in softwareontwikkeling was een belangrijke motivatie voor de keuze van dit bachelorproefonderwerp.\newline

Tijdens mijn opzoekwerk besefte ik dat de projecten waarmee ik tot dan toe in aanraking was gekomen voornamelijk gebaseerd waren op een monolithische architectuur. Naarmate ik meer leerde over modernere architecturen zoals microservices, begon ik me af te vragen of de traditionele monoliet wel de meest geschikte keuze is voor grootschalige applicaties. Microservices worden vaak genoemd in de context van schaalbare systemen, maar brengen ook nieuwe uitdagingen met zich mee. Hierdoor wilde ik beter begrijpen hoe beide architecturen zich in de praktijk van elkaar onderscheiden. Zo kwam ik op het idee om hun prestaties en codecomplexiteit tegenover elkaar te plaatsen in een concrete uitwerking.\newline

Deze bachelorproef had ik niet kunnen realiseren zonder de steun van enkele belangrijke personen. In het bijzonder wil ik mijn promotor, mevrouw Sonia Vandermeersch, bedanken voor haar waardevolle begeleiding en feedback gedurende de uitwerking van deze bachelorproef. Ook mijn co-promotor, de heer Chris Cokelaere, wil ik hartelijk bedanken voor zijn ondersteuning en betrokkenheid, zowel bij de bachelorproef als bij de stage.\newline

Tot slot dank ik u, beste lezer, om interesse te tonen en de tijd te nemen deze bachelorproef door te nemen.\newline

Vandenbroucke Jasper