%%=============================================================================
%% Inleiding
%%=============================================================================

\chapter{\IfLanguageName{dutch}{Inleiding}{Introduction}}%
\label{ch:inleiding}

\section{Context en achtergrond}

In de moderne softwareontwikkeling wordt steeds vaker gezocht naar schaalbare en onderhoudbare architecturen om de groeiende complexiteit van webapplicaties aan te kunnen. Traditioneel zijn heel wat applicaties ontwikkeld volgens een monolithische architectuur, waarbij alle functionaliteiten binnen één en dezelfde codebase beheerd worden. Hoewel een monolithische architectuur in de beginfase van een applicatie voordelen biedt, zoals eenvoudige ontwikkeling en implementatie, kan deze bij groeiende applicaties leiden tot problemen op het vlak van schaalbaarheid en onderhoudbaarheid.

Om deze problemen te voorkomen, wordt steeds vaker overgestapt naar een microservices\--architectuur. Deze architectuurstijl verdeelt een applicatie in meerdere kleine, onafhankelijke services, die elk een specifieke bedrijfsfunctionaliteit op zich nemen, zoals productenbeheer, bestellingen, gebruikers-/klantenbeheer enz. Deze aparte services communiceren met elkaar via API's (zoals REST of gRPC). Het grootste voordeel bij microservices is op het gebied van schaalbaarheid, flexibiliteit en onderhoudbaarheid, maar als kanttekening brengen ze een extra uitdaging met zich mee, zoals dataconsistentie en orkestratie.

Bij het stagebedrijf Fabrimode, dat actief is binnen de mode-industrie, is er interesse in het onderzoeken van de opties van microservices. Daarom zal in dit onderzoek de focus liggen op de potentiële voordelen van een overgang naar microservices. Het hoofddoel van deze bachelorproef is om inzicht te krijgen in de mogelijkheden van deze technologie voor Fabrimode.

\section{Probleemstelling}

Veel bedrijven ondervinden problemen bij het schalen van een monolithische applicatie, voornamelijk wanneer het aantal gebruikers toenemen. Een veelvoorkomend probleem is dat het moeilijk wordt om nieuwe functionaliteiten toe te voegen vanwege de codebase die steeds uitgebreider wordt.

Hierdoor ontstaat de vraag of een microservices-architectuur een aanzienlijke verbetering kan bieden op het gebied van schaalbaarheid in vergelijking met een traditionele, monolithische architectuur. Is de overstap naar microservices in dat geval wel zinvol?

Daarnaast brengt een microservices-architectuur ook uitdagingen met zich mee zoals:

\begin{itemize}
	\item Hoe presteert een microservices-architectuur in vergelijking met een monolithische architectuur onder verschillende omstandigheden van belasting?
	\item Is er extra complexiteit? Zo ja, hoe groot verschilt dit met een monolithische applicatie?
	\item Welke tools en technologieën zijn essentieel voor een geslaagde implementatie?
\end{itemize}

\section{Onderzoeksdoelstelling}

Biedt een microservices-architectuur een optimalisatie omtrent schaalbaarheid en welke impact heeft het op de ontwikkelingscomplexiteit? Dat is het doel van deze bachelorproef en zal aan de hand van een proof-of-concept (PoC) onderzocht worden.

De twee architecturen zullen geëvalueerd worden op basis van:

\begin{itemize}
	\item \textbf{Schaalbaarheid} - Hoe goed kunnen de applicaties omgaan met een groot aantal gebruikers?
	\item \textbf{Complexiteit} - Hoe beïnvloeden de verschillende architecturen de complexiteit?
\end{itemize}

De resultaten zullen Fabrimode helpen om te beslissen of een overgang naar microservices zinvol is en welke factoren er in acht genomen moeten worden.

\section{Onderzoeksvraag}

De onderzoeksvraag luidt als volgt:

\textit{Welke impact heeft een microservices-architectuur op de schaalbaarheid en ontwikkelingscomplexiteit van een webapplicatie?}

Volgende deelvragen zullen bijdragen tot de conclusie van deze bachelorproef:

\begin{itemize}
	\item Hoe presteert een microservices-architectuur in vergelijking met een monolithische architectuur op het gebied van schaalbaarheid?
	\item Welke aspecten moeten in acht genomen worden bij het overstappen naar een microservices-architectuur?
	\item Welke technologieën en tools zijn essentieel voor het implementeren en beheren van een microservices-architectuur?
\end{itemize}

\section{Methodologie}

Om de voorgaande vragen te beantwoorden zal er een proof-of-concept (PoC) uitgewerkt worden waarin twee veries van een e-commerce webapplicatie worden ontwikkeld:

\begin{enumerate}
	\item Monolithische applicatie:
	\begin{itemize}
		\item C\# en .NET 8
		\item Alles binnen één codebase
		\item Dataopslag via SQL Server-database
	\end{itemize}
	\item Microservices applicatie
	\begin{itemize}
		\item C\# en .NET 8
		\item Functionaliteiten opsplitsen in afzonderlijke services
		\item Communicatie via RabbitMQ
		\item Containerisatie via Docker en Kubernetes
		\item Elke service zijn eigen SQL Server-database
	\end{itemize}
\end{enumerate}

De beide applicaties zullen getest en geëvalueerd worden met behulp van twee tools:

\begin{itemize}
	\item \textbf{Apache JMeter} wordt gebruikt voor schaalbaarheidstesten, waarmee onderzocht zal worden hoe beide architecturen presteren onder verschillende belastingniveaus.
	\item Met de gratis versie van \textbf{SonarQube} zal een code-analyse uitgevoerd worden, waarbij gekeken wordt naar codecomplexiteit (cyclomatische complexiteit).
\end{itemize}

In hoofdstuk \ref{ch:methodologie} wordt dieper ingegaan op de methodologie en hoe de PoC opgebouwd is.

%De inleiding moet de lezer net genoeg informatie verschaffen om het onderwerp te begrijpen en in te zien waarom de onderzoeksvraag de moeite waard is om te onderzoeken. In de inleiding ga je literatuurverwijzingen beperken, zodat de tekst vlot leesbaar blijft. Je kan de inleiding verder onderverdelen in secties als dit de tekst verduidelijkt. Zaken die aan bod kunnen komen in de inleiding~\autocite{Pollefliet2011}:

%\begin{itemize}
%  \item context, achtergrond
%  \item afbakenen van het onderwerp
%  \item verantwoording van het onderwerp, methodologie
%  \item probleemstelling
%  \item onderzoeksdoelstelling
%  \item onderzoeksvraag
%  \item \ldots
%\end{itemize}

%\section{\IfLanguageName{dutch}{Probleemstelling}{Problem Statement}}%
%\label{sec:probleemstelling}

%Uit je probleemstelling moet duidelijk zijn dat je onderzoek een meerwaarde heeft voor een concrete doelgroep. De doelgroep moet goed gedefinieerd en afgelijnd zijn. Doelgroepen als ``bedrijven,'' ``KMO's'', systeembeheerders, enz.~zijn nog te vaag. Als je een lijstje kan maken van de personen/organisaties die een meerwaarde zullen vinden in deze bachelorproef (dit is eigenlijk je steekproefkader), dan is dat een indicatie dat de doelgroep goed gedefinieerd is. Dit kan een enkel bedrijf zijn of zelfs één persoon (je co-promotor/opdrachtgever).

%\section{\IfLanguageName{dutch}{Onderzoeksvraag}{Research question}}%
%\label{sec:onderzoeksvraag}

%Wees zo concreet mogelijk bij het formuleren van je onderzoeksvraag. Een onderzoeksvraag is trouwens iets waar nog niemand op dit moment een antwoord heeft (voor zover je kan nagaan). Het opzoeken van bestaande informatie (bv. ``welke tools bestaan er voor deze toepassing?'') is dus geen onderzoeksvraag. Je kan de onderzoeksvraag verder specifiëren in deelvragen. Bv.~als je onderzoek gaat over performantiemetingen, dan 

%\section{\IfLanguageName{dutch}{Onderzoeksdoelstelling}{Research objective}}%
%\label{sec:onderzoeksdoelstelling}

%Wat is het beoogde resultaat van je bachelorproef? Wat zijn de criteria voor succes? Beschrijf die zo concreet mogelijk. Gaat het bv.\ om een proof-of-concept, een prototype, een verslag met aanbevelingen, een vergelijkende studie, enz.

%\section{\IfLanguageName{dutch}{Opzet van deze bachelorproef}{Structure of this bachelor thesis}}%
%\label{sec:opzet-bachelorproef}

% Het is gebruikelijk aan het einde van de inleiding een overzicht te
% geven van de opbouw van de rest van de tekst. Deze sectie bevat al een aanzet
% die je kan aanvullen/aanpassen in functie van je eigen tekst.

%De rest van deze bachelorproef is als volgt opgebouwd:

%In Hoofdstuk~\ref{ch:stand-van-zaken} wordt een overzicht gegeven van de stand van zaken binnen het onderzoeksdomein, op basis van een literatuurstudie.

%In Hoofdstuk~\ref{ch:methodologie} wordt de methodologie toegelicht en worden de gebruikte onderzoekstechnieken besproken om een antwoord te kunnen formuleren op de onderzoeksvragen.

%% TODO: Vul hier aan voor je eigen hoofstukken, één of twee zinnen per hoofdstuk

%In Hoofdstuk~\ref{ch:conclusie}, tenslotte, wordt de conclusie gegeven en een antwoord geformuleerd op de onderzoeksvragen. Daarbij wordt ook een aanzet gegeven voor toekomstig onderzoek binnen dit domein.