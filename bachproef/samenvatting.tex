%%=============================================================================
%% Samenvatting
%%=============================================================================

% TODO: De "abstract" of samenvatting is een kernachtige (~ 1 blz. voor een
% thesis) synthese van het document.
%
% Een goede abstract biedt een kernachtig antwoord op volgende vragen:
%
% 1. Waarover gaat de bachelorproef?
% 2. Waarom heb je er over geschreven?
% 3. Hoe heb je het onderzoek uitgevoerd?
% 4. Wat waren de resultaten? Wat blijkt uit je onderzoek?
% 5. Wat betekenen je resultaten? Wat is de relevantie voor het werkveld ?
%
% Daarom bestaat een abstract uit volgende componenten:
%
% - inleiding + kaderen thema
% - probleemstelling
% - (centrale) onderzoeksvraag
% - onderzoeksdoelstelling
% - methodologie
% - resultaten (beperk tot de belangrijkste, relevant voor de onderzoeksvraag)
% - conclusies, aanbevelingen, beperkingen
%
% LET OP! Een samenvatting is GEEN voorwoord!

%%---------- Nederlandse samenvatting -----------------------------------------
%
% TODO: Als je je bachelorproef in het Engels schrijft, moet je eerst een
% Nederlandse samenvatting invoegen. Haal daarvoor onderstaande code uit
% commentaar.
% Wie zijn bachelorproef in het Nederlands schrijft, kan dit negeren, de inhoud
% wordt niet in het document ingevoegd.

%\IfLanguageName{english}{%
%\selectlanguage{dutch}
%\chapter*{Samenvatting}
%\lipsum[1-4]
%\selectlanguage{english}
%}{}

%%---------- Samenvatting -----------------------------------------------------
% De samenvatting in de hoofdtaal van het document

\chapter{Samenvatting}

De impact van een microservices-architectuur op de schaalbaarheid en complexiteit van webapplicaties staat centraal in deze bachelorproef. Het stagebedrijf Fabrimode wil weten of een overstap naar microservices de prestaties van een webapplicatie kunnen verbeteren. In de hedendaagse softwareontwikkeling, waarin schaalbaarheid en onderhoudbaarheid van cruciaal belang zijn, biedt de microservices-architectuur een alternatief door applicaties op te splitsen in onafhankelijke services. Daarom luidt de onderzoeksvraag als volgt: "Welke impact heeft een microservices\--architectuur op de schaalbaarheid en ontwikkelingscomplexiteit van een webapplicatie?".\newline

De monolithische architectuur ondervindt vaak schaalbaarheidsproblemen bij groei, terwijl microservices een mogelijke oplossing bieden voor dit probleem, maar mogelijks de ontwikkelingscomplexiteit verhogen. Het doel van deze bachelorproef is om na te gaan of microservices een optimalisatie bieden op het gebied van schaalbaarheid en complexiteit. Dit zal bekeken worden aan de hand van een proof-of-concept (PoC), die in hoofdstuk \ref{ch:methodologie} verder wordt toegelicht.\newline

De methodologie die zal toegepast worden bestaat in eerste instantie uit een literatuurstudie, waarin de verschillen tussen beide architecturen nader bekeken zullen worden. Volgend op de literatuurstudie is het ontwikkelen van een proof-of-concept (PoC) met twee versies van een e-commerce webapplicatie (monolithisch en microservices). Beide zullen ontwikkeld worden in C\# met .NET 8. Daarnaast zal voor het implementeren en beheren van microservices gebruik gemaakt worden van Docker en Kubernetes samen met RabbitMQ als communicatie tool. Tot slot zullen na het uitwerken van de PoC de schaalbaarheid en complexiteit getest en geanalyseerd worden met respectievelijk Apache JMeter en SonarQube als tools.\newline
